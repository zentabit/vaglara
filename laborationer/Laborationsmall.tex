% Laborationsmall.tex
\documentclass[a4paper]{article}

\usepackage[swedish]{babel}
\usepackage[utf8x]{inputenc}

\usepackage{multicol}
\usepackage[vmargin=3cm,hmargin=2cm]{geometry}
\usepackage{parskip}
\usepackage[runin]{abstract}
\renewcommand{\abstitleskip}{0mm}

\usepackage{lmodern}
\usepackage[T1]{fontenc}

\usepackage{graphicx}
\usepackage{ccaption}
\captionnamefont{\it}
\captiontitlefont{\it}

% Hack för att få komma istället för punkt i matematiska uttryck
% $3.141592$ blir 3,141592
% Om man använder komma direkt får man ett litet oönskat mellanrum:
% $3,141592$ blir 3, 141592
\DeclareMathSymbol{,}{\mathpunct}{letters}{"3B}
\DeclareMathSymbol{.}{\mathord}{letters}{"3B}
\DeclareMathSymbol{\decimal}{\mathord}{letters}{"3A}

% Kommando för att få icke-kursiva enheter i matematiska uttryck
% $10\unit{km}$ blir 10 km
\newcommand{\unit}[1]{\ensuremath{\,\mathrm{#1}}}

\usepackage{lastpage}
\usepackage{fancyhdr}
\pagestyle{fancy}
\fancyhf[C]{\thepage}
\fancyhead[C]{Våglära och optik, FAFF30}
\fancyhead[R,L]{}
\fancypagestyle{plain}{
  \fancyhead{}
}
\setcounter{secnumdepth}{-1}

\title{Laborationen ”Polarisation”}
\author{Johan Mauritsson\\Lunds Universitet}

\makeatletter
\renewcommand{\section}{\@startsection
{section}%                   % the name
{1}%                         % the level
{0mm}%                       % the indent
{-\baselineskip}%            % the before skip
{0.5mm}%          % the after skip
{\normalfont\bfseries}} % the style
%\renewcommand{\sectionmark}[1]{ }
%\renewcommand{\thesection}{}

\renewcommand*\maketitle{
  {
    \begin{center}
      {\huge\bfseries \@title}\\
      \vspace{5mm}
      {\large \@author}
    \end{center}
    \vspace{2mm}
  }
}
\makeatother

\begin{document}
\maketitle


\begin{abstract}
  Kort beskrivning  av resultaten (5-10  rader). Det ska inte  vara en
  innehållsbeskrivning (först gör vi det, sen använder vi den metoden,
  och så jämför vi det med de där tidigare kända resultaten, etc) utan
  vara   koncentrerat  till   ”resultatet”,   vad   man  kommer   fram
  till. Sammanfattningen är uppsatsens löpsedel. Den ska vara kort och
  kärnfull och locka läsare genom att  effektivt göra klart vad det är
  man uppnår genom att läsa rapporten.
\end{abstract}

\vspace{2mm}

\hspace{-3mm}
\begin{tabular}{ll}
Laborationen genomförd: &	2013-xx-xx \\
Rapporten kamratgranskad: &	2013-xx-xx \\
Lämnad till handledare: &	2013-xx-xx \\
\end{tabular}

\vspace{3mm}

\begin{multicols}{2}
  \section*{Inledning}
  Preliminär beskrivning av uppgiften  i lättfattliga termer. Bakgrunden
  till att man intresserar sig för detta problem, denna uppgift. Hur den
  ingår i ett större sammanhang.

  En detaljerad beskrivning av vad det hela går ut på, och vad exakt din
  uppgift   i    sammanhanget   är.    Utrustningsförutsättningar.   Det
  ursprungligen avsedda målet. Även den i ämnet oinsatte ska ha en ärlig
  chans att förstå de stora dragen.

  Beskrivning av hur rapporten är uppbyggd, för att göra det möjligt för
  läsaren att förstå vad som  pågår, ge läsaren rätta förväntningar, och
  för  att  underlätta direktåtkomst  av  för  den individuelle  läsaren
  särskilt intressanta delar.

  \section{Metod}
  De  olika  stegen  i  uppgiftens genomförande.  Till  exempel  val  av
  algoritmer,  programspråk och  annan programvara,  undersökningsmetod,
  statistiska  metoder. Där  valmöjligheter finns,  diskutera de  gjorda
  valen.

  \section{Resultat}
  Vad man  kommer fram till?  Om uppgiften innebär programmering  så kan
  det färdiga programmet i sig vara resultatet.  Här är det i så fall på
  sin  plats  med  beskrivningar  av programmet,  till  exempel  teknisk
  beskrivning,    funktionell    beskrivning,    användningsbeskrivning,
  användargränssnitt  etc.  Genomförandebeskrivning  kommer  i  så  fall
  närmast att handla om programutvecklingen.

  Men det kan också vara så  att det snarare är programmets beteende som
  är resultatet.  Man utvecklar till  exempel en numerisk metod  för att
  lösa ett problem,  och sedan gör man experiment  med testkörningar och
  analyserar dessa test.

  \section{Diskussion}
  Är resultaten rimliga? Vad hade kunnat göras annorlunda?

  \section{Slutsats/avslutning}
  En   sammanfattning  där   man  till   skillnad  från   den  inledande
  sammanfattningen förutsätter  att läsaren har läst  rapporten, samt de
  slutsatser man kan dra av det gjorda arbetet.

  \section{Referenser}
  Den använda litteraturen.

  \section{Appendix}
  Eventuell     användarhandledning,     källkod,    anvisningar     för
  systemgenerering, och liknande som inte  är av omedelbart intresse för
  den ordinäre läsaren, läggs lämpligen som appendix.

\end{multicols}

\section{\LaTeX-kommentarer}
Kompileras enklast med \texttt{pdflatex}. Använd TeX Live i Linux och
MiKTeX i Windows.

Test av ”komma-hack” och enhetskommando (se källkod):
\hrule
\begin{minipage}{0.5\linewidth}
  Kod
\begin{verbatim}
\[ \Delta x = 1.23\unit{km} \]
\end{verbatim}
\end{minipage}
\begin{minipage}{0.5\linewidth}
  Resultat\\
  $ \Delta x = 1.23\unit{km} $
\end{minipage}

\end{document}
