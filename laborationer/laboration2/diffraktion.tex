\documentclass[a4paper]{article}
\usepackage[a4paper, total={6in, 9in}]{geometry}
\usepackage{amsmath}
\usepackage{csvsimple}
\usepackage{graphicx}
\usepackage{siunitx}
\usepackage{float}
\usepackage{subcaption}
\usepackage{blindtext}
\usepackage[hidelinks]{hyperref}
\usepackage[style=iso]{datetime2}

\graphicspath{ {./img/} }

\title{Sammanfattning av Diffraktionslaboration}

\author{Emil Babayev}
\date{\today}
\begin{document}
\begin{titlepage}
	\centering
	\includegraphics[width=0.15\textwidth]{logo.png}\par\vspace{1cm}
	{\scshape\large Lunds Tekniska Högskola \par}
	\vspace{1cm}
    {\scshape\large FAFF30 Våglära och optik\par}
	\vspace{1.5cm}
	{\huge\bfseries Sammanfattning\\Diffraktionslaboration\par}
	\vspace{2cm}
	{\Large Emil Babayev\par}
	\vfill
	Laborationshandledare\par
    Samuel Bengtsson

    \vfill
    
	{\large \today \par}
\end{titlepage}

\section{Inledning}
Länge har det diskuterats om ljuset har våg- eller partikelnatur. Diskussionen har pågått i flera hundra år och man är fortfarande inte helt överens kring vilket det är,
utan accepterar istället att ljuset uppvisar både våg- och partikelegenskaper. I denna laboration har fenomenet diffraktion(även känt som böjning) studerats, vilket är 
ett typiskt vågfenomen. Det innebär att alla resultat som åstadkommits i laborationen tyder på att ljus iallafall definitivt är en våg. Alla uppställningar har gjorts med en
röd, koherent, monokromatisk laser.

\section{Sammanfattning av laborationen}
\subsection{Fraunhoferdiffraktion}
\subsubsection{Dubbelspalt}

\begin{table}[h!]
	\begin{tabular}{|l|l|l|l|}
	\hline
	Maximats ordning (m) & Avstånd till maxima (x/cm) & Avstånd till skärmen (L/cm) & \begin{tabular}[c]{@{}l@{}}Beräknat spaltavstånd\\  (d/mm)\end{tabular} \\ \hline
	1                    & 0,24                       & 99,775                      & 0,2666                                                                  \\ \hline
	2                    & 0,5                        & 99,775                      & 0,2559                                                                  \\ \hline
	3                    & 0,65                       & 99,775                      & 0,2953                                                                  \\ \hline
	\end{tabular}
\end{table}

\subsubsection{Enkelspalt}

\begin{table}[h!]
	\begin{tabular}{|l|l|l|l|}
	\hline
	Minimats ordning (m) & Avstånd till minima (x/cm) & Avstånd till skärmen (L/cm) & \begin{tabular}[c]{@{}l@{}}Beräknat spaltbredd\\  (d/mm)\end{tabular} \\ \hline
	1                    & 0,325                      & 99,9                        & 0,1973                                                                \\ \hline
	2                    & 0,65                       & 99,9                        & 0,1973                                                                \\ \hline
	3                    & 0,95                       & 99,9                        & 0,2025                                                                \\ \hline
	4                    & 1,3                        & 99,9                        & 0,1973                                                                \\ \hline
	\end{tabular}
\end{table}

\subsubsection{Babinets princip}
\subsubsection{Flerspaltsystem}
\subsection{Fresneldiffraktion}
\subsubsection{Ställbar spalt}
\subsubsection{Rak kant}
\subsubsection{Hål}
\begin{table}[h!]
	\begin{tabular}{|l|l|l|}
	\hline
	Fresnelzon (fasskillnad $\lambda$/m) & Avstånd till hålet (L/cm) & Beräknad hålradie (r/mm) \\ \hline
	1                                 & 7,76                      & 0,3153                   \\ \hline
	1,5                               & 5,05                      & 0,3117                   \\ \hline
	2                                 & 3,6                       & 0,2996                   \\ \hline
	\end{tabular}
	\end{table}
\subsubsection{Aragos fläck}
\section{Referenser och appendix}
\end{document}
